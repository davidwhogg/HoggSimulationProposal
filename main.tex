\documentclass[12pt]{article}
\usepackage[letterpaper]{geometry}
\usepackage{setspace}

% page layout
\setstretch{1.08}
\setlength{\topmargin}{0in}
\setlength{\headheight}{2ex}
\setlength{\headsep}{4ex}
\setlength{\textheight}{9in}\addtolength{\textheight}{-\headheight}\addtolength{\textheight}{-\headsep}
\setlength{\textwidth}{6.5in}
\setlength{\oddsidemargin}{0.5\paperwidth}\addtolength{\oddsidemargin}{-1.0in}\addtolength{\oddsidemargin}{-0.5\textwidth}
\pagestyle{myheadings}
\markboth{foo}{\sffamily Hogg / Does the simulation hypothesis lead to physics questions?}
\renewcommand{\newblock}{} % this adjusts the bibliography style.
\frenchspacing\sloppy\sloppypar\raggedbottom

\begin{document}

\subsubsection*{Title}

\subsubsection*{Executive Summary (1,300 Character)}
% The Executive Summary should briefly address the following questions:
% (a) What specific questions will your project answer?
% (b) What activities will you carry out to answer those questions?
% (c) Why is this project needed?
% (d) What concrete deliverables will you produce by the end of the project?
% (e) What impact will your project have?

\noindent\textsl{(a)}~This project is to consider the experimental-physics consequences of the possibility that we might be living inside some kind of simulation.
The project will develop some empirical predictions that flow from various (of myriad) possible simulation hypotheses. The project will also produce a categorization or classification of simulation hypotheses along a set of binary considerations.

\smallskip
\noindent\textsl{(b)}~This is a theoretical-physics project, which will support the PI, a PhD student at NYU, and undergraduate researchers. The activities include writing several papers for the refereed cosmology or theoretical physics literatures. The investigator and project personnel will also give seminars and attend conferences.

\smallskip
\noindent\textsl{(c)}~Most of the many possible simulation hypotheses are, fundamentally, cosmological hypotheses, but they are not taken seriously by the physics community, because of certain faiths about physical law. Simulation hypotheses make predictions for experiments in much the same way that searches for new physics make predictions. They are the only experimental-physics directions that could ever come close to finding evidence for something akin to ``our maker.''

\smallskip
\noindent\textsl{(d)}~The deliverables will be our papers and our seminars.

\smallskip
\noindent\textsl{(e)}~If we do indeed live in a simulation (though unlikely \textsl{a priori}), that would be by far the most important discovery in the history of physics. Even if we never find any such evidence, this project will bring the simulation hypothesis into the domain of physical cosmology.

\subsubsection*{Project Description: (4,000 Character)}
Science fiction movies---including the classics \textit{Total Recall} (1990) and \textit{The Matrix} (1999) from my youth---abound with different variants of the simulation hypothesis.
The \emph{simulation hypothesis} is the hypothesis that the world or universe that we inhabit is in fact a simulation, and that we are experiencing a simulated reality.
The simulation hypothesis has obtained some interest from philosophers (most notably \cite{bostrom}), but the work is very conceptual, and fails to fit within some basic physical constraints.
The question posed in this grant proposal is the following:
\textbf{Is there any sense in which the simulation hypothesis generates testable predictions?}
A related question might be:
\textbf{Is the simulation hypothesis a hypothesis in physical cosmology?}
I will argue that it appears that the answer to both of these questions is ``yes.''
However, this proposal is primarily to develop more specific versions of the relevant questions, and not so much to answer them.

Before we start, I will warn the reader that the simulation hypothesis makes testable predictions in very much the same way that searches for new physics make testable predictions:
When we search for new physics, we search for generic but plausible modifications or perturbations to our fundamental theories,
using experimental tools and data that are available to us.
When we do not find a signal for which we have searched, we do not conclude ``there is no new physics here.''
We develop quantitative upper limits on the proposed term or modification or perturbation.
That is, the search for new physics proceeds by creatively generating hypotheses, testing them, and setting upper limits.
In the rare cases in which positive evidence is found for new physics, it is extremely exciting; these cases lead to additional hypotheses and tests.
However, in the last decades of physics, almost none have held up (neutrino oscillations notwithstanding; most would argue that the non-trivial mass matrix was not actually new physics, just previously unobserved parameters) to those subsequent tests.

Similarly, the simulation hypothesis will suggest various kinds of generic possible modifications to the laws of physics or to what is observable, physically.
The goal of this research project is to begin the process of getting those ideas written down and to start the process of hypothesizing and testing the easiest of the testable hypotheses.
Since any simulator or simulation can hide arbitrarily well---since any sufficiently adversarial simulation will be impossible to detect at any finite precision---and since it is also perfectly reasonable that we don't (in fact) live in any kind of  simulation, I expect that most of these tests will provide only upper limits or constraints on particular simulation properties.
That said, the most ambitious part of this proposal is \textbf{to demonstrate that the hypothesis can lead to quantitative experimental-physics projects}.

The most achievable part of this proposal is the establishment of \textbf{a classification of simulation hypotheses}.
There are many concepts of what a simulation might be.
For example, any Hamiltonian system (classical or quantum) is exactly quantitatively analogous to any other Hamiltonian system (classical or quantum, respectively) of the same dimensionality; indeed all such systems are related by (possibly exceedingly complex) canonical transformations.
Would that constitute a simulation? Or does a simulation also need to be intentionally \emph{created by a simulator}?
In some simulation hypotheses, this universe is being simulated with the express purpose of understanding humans; in other simulation hypotheses, the emergence of humans is incidental and unintended.
Some simulations are being actively observed or even manipulated; others are left to their own devices.
Some simulators are very careful to conceal to the simulation inhabitants that the inhabitants are in fact in a simulation; others are more careless.
Each of these kinds of considerations leads to a (binary, perhaps) classification; the union of all the considerations produces a detailed classification with some kinds of completeness.

One amusing outcome of this research direction (and I should emphasize that this is \emph{not} my main goal) is that an empirical study of the simulation hypothesis is in some very real sense an empirical search for some kind of God or maker:
If an empirical study could show that we appear to be in an intentionally created universe, then in some sense that study delivers evidence for the existence of the maker of this universe.
This not my own primary motivation, but this consequence does make this project very appropriate for funding by the Templeton Foundation.

What \emph{is} then, my primary motivation?
From my perspective, the simulation hypothesis is a subject of great interest to students of physics and the general public, but which is actively rejected or scorned by most professional physicists.
When I teach NYU~Physics~I or II for the physics majors, they come into these first classes with a sense of wonder about many things, including the simulation hypothesis.
Indeed, I have a period in my classes in which students can ``ask me anything'' and \emph{literally every semester} one of the physics majors has asked me ``Do we live in a simulation?''
The professionalization of physics trains them to stop asking such ridiculous questions!
But I, for one, do not think this question is ridiculous.
I think this is a question in the subject of physical cosmology.
And I think most physicists' rejection of this question is based on prejudice (or \emph{faith} if you will) not empirical data.
This proposal is to start a new conversation in the physics community, a conversation that undergraduate pre-physicists, and people of myriad backgrounds, are eager to have.

It's extremely unlikely (in some absolute sense) that we live in a simulation.
However, if we do, the discovery would be the biggest possible discovery in the history of physics.

\paragraph{Classification of simulation hypotheses:}
The start of this project is the establishment of a classification.
This is subject to change depending on what we find, but currently my thought is that the classification will be a set of binary switches or yes--no questions.
To give an idea of where this is going, here are some possible binaries, with discussion:

\textsl{Is the simulation exact, or approximate?}

\textsl{Is the simulation aimed at understanding humans?}

\textsl{Is the simulation being actively observed?}

\textsl{Is the simulation hiding?}

\textsl{Is the simulation digital or analog?}

In addition to these there are many more binary switches (quantum or classical? intervened in or left alone? simulation within a simulation? and so on).
The proposed activity is to enumerate them, complete the classification, and ask which leaves of the binary tree produced by the classification deliver observational consequences.
Some of them will deliver such consequences; this proposal is to follow up and make some relevant predictions or design some searches for the ``new physics'' so generated.

\paragraph{Notes:}
Here are some random thoughts that might be of relevance to this proposal:
\begin{itemize}
\item Feynman brought up the point that one quantum simulation can precisely emulate any other quantum system; it is possible to do exact quantum simulation.
\item Feynman brought up the question of whether time and space are quantized.
\item If a hamiltonian system is being observed, unitary must be broken at some level.
\item In our own Universe, unitarity is broken at the horizons of black holes. Is this Universe just a black-hole accretion simulation and it just happens to be so high in resolution that it produced the Earth and us?
\item The Bostrom argument is wrong because of computing capacity.
\item If there is an approximate simulation, it might have a coordinate system, and thus our world wouldn't be precisely coordinate free.
\item If there is an approximate simulation, the approximation might lead to thresholds or jumps in something.
\item The simplicity of the CDM model is sort-of evidence that we live in a Earth-centered or human-centered or local-Universe-centered simulation.
\item In the case that the simulation is Earth centered, we really do live in the center of the Universe.
\item If the universe is an approximate simulation, then there is no reason for large-scale and small-scale physics models to be compatible. Hahahaha.
\item If the universe is being simulated but only incidentally, since humans are the objects of study, then the physics of the universe would probably be very simple. Hahaha.
\item Bring up the \textsl{a posteriori} aspects of this.
\end{itemize}

\subsubsection*{Statement of Significance (1,300 Character)}
% Describe the current conditions in the field(s) relevant to the project, identify the problems that the project will address, and articulate the specific opportunity that your project presents.
Right now, the field of study for this proposal more-or-less doesn't exist, despite relevant papers by Feynman many years ago.
Almost no research-active physicist takes seriously the possibility that we live in a simulation, and, if asked, would say that the hypothesis makes no predictions anyway.
Those responses are glib, however:
Simulation hypotheses make predictions in the same way that all searches for new physics make predictions.

There are generic simulation-relevant observables; if we don't see anything in those channels, we can place limits.
No individual observation can rule out all new physics, and new physics can certainly hide from all experimental techniques, in just the same way that a simulation can hide from any experiment.
If we do live in a simulation, it is undeniably important, from a cosmological and humanistic perspective, to know that.
This proposal is to start the physics community thinking about simulation hypotheses as (in many cases) experimentally testable physical hypotheses.

More specifically, there are some conversations in philosophy about the simulation hypothesis.
The most famous of these (by Bostrom) makes a basic, fundamental flaw in its physical assumptions.
Since this paper has been influential in philosophy and in the public mind, it is very much worth correcting.

\subsubsection*{Outputs (1,300 Character)}
% Outputs (sometimes called "deliverables") are important events and work products that your Project activities (described above) will lead to, and which are necessary in order for you to make progress towards your proposed Outcomes. Please provide a list of the outputs you intend to produce.
The specific scientific goal of this proposal is achievable and sensible:
The plan is to write a scientific papers for the refereed literature (physics or cosmology).
The first of these will include a classification of all simulation hypotheses or scenarios.
Subsequently, the plan is to develop the observational consequences of the more observable scenarios, and write a paper or several short papers on these consequences.
Another deliverable will be a paper for the philosophy literature commenting on and refuting the argument of Bostrom \cite{bostrom}; this is an influential paper about the simulation hypothesis but wrong on physical-computation grounds.
That is, this proposal is to do normal science, but in the context of an extraordinary subject.

The project will also develop a set of scientific seminars, for audiences ranging from amateur scientists, to undergraduate physics majors, to professional cosmologists; and to give those seminars at relevant locations in New York City and in the world.
These seminars will be primarily physics seminars, but they will make connections to philosophy and the arts, because this subject connects to many ideas of importance outside physics.

In terms of personnel, this proposal will support one graduate student research associate (GRA), who will develop the theoretical physics aspects of the proposal.
It will also support one undergraduate or post-bacc researcher, who will develop the literature review and the connections to ideas in philosophy, physics, and the arts.
It will also support a small fraction of my annual salary, plus some travel and a laptop computer for the GRA.

\subsubsection*{Outcomes (1,300 Character)}
% Outcomes (sometimes called goals, results, or impacts) are the specific and identifiable changes that you expect your Outputs will bring about (or contribute to bringing about) within 5 years of your project's end date. These should describe what the success of your project would look like. Please provide a list of the outcomes you expect to come about as a result of your outputs.
The most important outcome of this work would be to get the physics community thinking about what might be testable consequences of any simulation hypotheses, and perhaps inspire real experimental work.
Another outcome will be to inspire those who are confident that we \emph{don't} live in a simulation to start the process of putting this belief on a sound observational or empirical setting.
The most unlikely---but highest conceivable impact---outcome is some experimental findings that do suggest that we are in a simulation.
That's not an expected outcome, but it's a conceivable one!

\subsubsection*{Capacity for Success (1,300 Character)}
% Explain why your team and/or organization is positioned to be successful in this project.
This is a departure for the PI; it is a departure for any research-active physicist!
The PI, however, has done work in fundamental physical cosmology over the years.
He is a co-discoverer of the baryon acoustic oscillations, which are the basis for much of precison cosmology.
He has measured the fractal dimension of the large-scale structure in the Universe and tested cosmic homogeneity and isotropy.
He is currently working on projects to comprehensively find all deviations of the standard model of physical cosmology visible in the cosmic microwave background and in the large-scale structure.
Recently he has made the most precise measurements ever of the local density of dark matter, and has a career-long project to measure the dark matter in collapsed galaxy halos as precisely as possible given astronomical sources and tools.
Thus, although the PI has not worked on the simulation hypothesis, the PI has worked extensively in the use of cosmology and astronomy observations to test fundamental theories.

The PI's home institution is the Center for Cosmology and Particle Physics in the Department of Physics at NYU.
Searches for new physics are a major research direction in the Center.
Thus the GRA on the project will benefit not just from the PI's mentorship but also input and support from a substantial number of faculty and many peers working on new physics.

\subsubsection*{Relation to Sir John Templeton's Donor Intent (1,000 Character)}
% To learn more about the Foundations Funding Areas please visit our Funding Areas page.
This project is very closely related to the Sir John Templeton's intent.
It is \textsl{a priori} unlikely that we live in a simulation!
However, if we do, that discovery would \emph{literally be the discovery of our maker}.
That is, this is one of the few conceivable research programs that uses physics experiments to ask whether there is concrete evidence in this universe of design and intervention.

\subsubsection*{Project Relationship to Previous Grants}
% To the best of your knowledge, is the work of your proposed project similar to, a continuation of, or an expansion of an active or completed grant you or your organization received from either the John Templeton Foundation, the Templeton Religion Trust, or the Templeton World Charity.
This project is unrelated to any funded projects led by the PI;
the PI primarily works in precision measurement and computational data analysis; he has received NASA and NSF grants for his research, and support over the years from the Sloan, Moore, and Simons Foundations.

\subsubsection*{CVs of project leader and co-leader}

\subsubsection*{History with the Foundation (1,000 Character Limit)}
% Please describe how you or any members of your team came to learn about the Foundation, including past grants, participation in Foundation-sponsored events, and/or discussions with staff about the project idea.
The only connection of the PI to the Foundation is that he served on a Foundation review panel (panel chair Tegmark) in 2001 or thereabouts.

\subsubsection*{Listing all personnel and organizational affiliation of each}
David W. Hogg, New York University; unnamed GRA, New York University; unnamed undergraduate researcher, New York University

\end{document}
